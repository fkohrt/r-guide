% Options for packages loaded elsewhere
\PassOptionsToPackage{unicode}{hyperref}
\PassOptionsToPackage{hyphens}{url}
%
\documentclass[
]{book}
\title{A Succinct Intro to R}
\author{Steve Haroz}
\date{2021}

\usepackage{amsmath,amssymb}
\usepackage{lmodern}
\usepackage{iftex}
\ifPDFTeX
  \usepackage[T1]{fontenc}
  \usepackage[utf8]{inputenc}
  \usepackage{textcomp} % provide euro and other symbols
\else % if luatex or xetex
  \usepackage{unicode-math}
  \defaultfontfeatures{Scale=MatchLowercase}
  \defaultfontfeatures[\rmfamily]{Ligatures=TeX,Scale=1}
\fi
% Use upquote if available, for straight quotes in verbatim environments
\IfFileExists{upquote.sty}{\usepackage{upquote}}{}
\IfFileExists{microtype.sty}{% use microtype if available
  \usepackage[]{microtype}
  \UseMicrotypeSet[protrusion]{basicmath} % disable protrusion for tt fonts
}{}
\makeatletter
\@ifundefined{KOMAClassName}{% if non-KOMA class
  \IfFileExists{parskip.sty}{%
    \usepackage{parskip}
  }{% else
    \setlength{\parindent}{0pt}
    \setlength{\parskip}{6pt plus 2pt minus 1pt}}
}{% if KOMA class
  \KOMAoptions{parskip=half}}
\makeatother
\usepackage{xcolor}
\IfFileExists{xurl.sty}{\usepackage{xurl}}{} % add URL line breaks if available
\IfFileExists{bookmark.sty}{\usepackage{bookmark}}{\usepackage{hyperref}}
\hypersetup{
  pdftitle={A Succinct Intro to R},
  pdfauthor={Steve Haroz},
  hidelinks,
  pdfcreator={LaTeX via pandoc}}
\urlstyle{same} % disable monospaced font for URLs
\usepackage{color}
\usepackage{fancyvrb}
\newcommand{\VerbBar}{|}
\newcommand{\VERB}{\Verb[commandchars=\\\{\}]}
\DefineVerbatimEnvironment{Highlighting}{Verbatim}{commandchars=\\\{\}}
% Add ',fontsize=\small' for more characters per line
\usepackage{framed}
\definecolor{shadecolor}{RGB}{248,248,248}
\newenvironment{Shaded}{\begin{snugshade}}{\end{snugshade}}
\newcommand{\AlertTok}[1]{\textcolor[rgb]{0.94,0.16,0.16}{#1}}
\newcommand{\AnnotationTok}[1]{\textcolor[rgb]{0.56,0.35,0.01}{\textbf{\textit{#1}}}}
\newcommand{\AttributeTok}[1]{\textcolor[rgb]{0.77,0.63,0.00}{#1}}
\newcommand{\BaseNTok}[1]{\textcolor[rgb]{0.00,0.00,0.81}{#1}}
\newcommand{\BuiltInTok}[1]{#1}
\newcommand{\CharTok}[1]{\textcolor[rgb]{0.31,0.60,0.02}{#1}}
\newcommand{\CommentTok}[1]{\textcolor[rgb]{0.56,0.35,0.01}{\textit{#1}}}
\newcommand{\CommentVarTok}[1]{\textcolor[rgb]{0.56,0.35,0.01}{\textbf{\textit{#1}}}}
\newcommand{\ConstantTok}[1]{\textcolor[rgb]{0.00,0.00,0.00}{#1}}
\newcommand{\ControlFlowTok}[1]{\textcolor[rgb]{0.13,0.29,0.53}{\textbf{#1}}}
\newcommand{\DataTypeTok}[1]{\textcolor[rgb]{0.13,0.29,0.53}{#1}}
\newcommand{\DecValTok}[1]{\textcolor[rgb]{0.00,0.00,0.81}{#1}}
\newcommand{\DocumentationTok}[1]{\textcolor[rgb]{0.56,0.35,0.01}{\textbf{\textit{#1}}}}
\newcommand{\ErrorTok}[1]{\textcolor[rgb]{0.64,0.00,0.00}{\textbf{#1}}}
\newcommand{\ExtensionTok}[1]{#1}
\newcommand{\FloatTok}[1]{\textcolor[rgb]{0.00,0.00,0.81}{#1}}
\newcommand{\FunctionTok}[1]{\textcolor[rgb]{0.00,0.00,0.00}{#1}}
\newcommand{\ImportTok}[1]{#1}
\newcommand{\InformationTok}[1]{\textcolor[rgb]{0.56,0.35,0.01}{\textbf{\textit{#1}}}}
\newcommand{\KeywordTok}[1]{\textcolor[rgb]{0.13,0.29,0.53}{\textbf{#1}}}
\newcommand{\NormalTok}[1]{#1}
\newcommand{\OperatorTok}[1]{\textcolor[rgb]{0.81,0.36,0.00}{\textbf{#1}}}
\newcommand{\OtherTok}[1]{\textcolor[rgb]{0.56,0.35,0.01}{#1}}
\newcommand{\PreprocessorTok}[1]{\textcolor[rgb]{0.56,0.35,0.01}{\textit{#1}}}
\newcommand{\RegionMarkerTok}[1]{#1}
\newcommand{\SpecialCharTok}[1]{\textcolor[rgb]{0.00,0.00,0.00}{#1}}
\newcommand{\SpecialStringTok}[1]{\textcolor[rgb]{0.31,0.60,0.02}{#1}}
\newcommand{\StringTok}[1]{\textcolor[rgb]{0.31,0.60,0.02}{#1}}
\newcommand{\VariableTok}[1]{\textcolor[rgb]{0.00,0.00,0.00}{#1}}
\newcommand{\VerbatimStringTok}[1]{\textcolor[rgb]{0.31,0.60,0.02}{#1}}
\newcommand{\WarningTok}[1]{\textcolor[rgb]{0.56,0.35,0.01}{\textbf{\textit{#1}}}}
\usepackage{longtable,booktabs,array}
\usepackage{calc} % for calculating minipage widths
% Correct order of tables after \paragraph or \subparagraph
\usepackage{etoolbox}
\makeatletter
\patchcmd\longtable{\par}{\if@noskipsec\mbox{}\fi\par}{}{}
\makeatother
% Allow footnotes in longtable head/foot
\IfFileExists{footnotehyper.sty}{\usepackage{footnotehyper}}{\usepackage{footnote}}
\makesavenoteenv{longtable}
\usepackage{graphicx}
\makeatletter
\def\maxwidth{\ifdim\Gin@nat@width>\linewidth\linewidth\else\Gin@nat@width\fi}
\def\maxheight{\ifdim\Gin@nat@height>\textheight\textheight\else\Gin@nat@height\fi}
\makeatother
% Scale images if necessary, so that they will not overflow the page
% margins by default, and it is still possible to overwrite the defaults
% using explicit options in \includegraphics[width, height, ...]{}
\setkeys{Gin}{width=\maxwidth,height=\maxheight,keepaspectratio}
% Set default figure placement to htbp
\makeatletter
\def\fps@figure{htbp}
\makeatother
\setlength{\emergencystretch}{3em} % prevent overfull lines
\providecommand{\tightlist}{%
  \setlength{\itemsep}{0pt}\setlength{\parskip}{0pt}}
\setcounter{secnumdepth}{5}
\usepackage{booktabs}
\usepackage{fontspec}
\setmainfont{Segoe UI}
\ifLuaTeX
  \usepackage{selnolig}  % disable illegal ligatures
\fi
\usepackage[]{natbib}
\bibliographystyle{plainnat}

\begin{document}
\maketitle

{
\setcounter{tocdepth}{1}
\tableofcontents
}
\hypertarget{about}{%
\chapter*{About}\label{about}}
\addcontentsline{toc}{chapter}{About}

This book is a short introduction to the R language. It covers the basics of R that are not covered by analysis and visualization guides like \href{https://r4ds.had.co.na}{R for Data Science}. Consider it a quick way to get up to speed on R before diving into the analysis and visualization aspects.

This example-focused guide assumes you are familiar with programming concepts but want to learn the R language. It offers more examples than an ``R cheat sheet'' without the verbosity of a language spec or an introduction to programming.

\hypertarget{sources-of-inspiration}{%
\section*{Sources of inspiration}\label{sources-of-inspiration}}

\url{http://alyssafrazee.com/introducing-R.html}

\href{http://www.johndcook.com/R_language_for_programmers.html}{R for programmers}

\hypertarget{acknowledgements}{%
\section*{Acknowledgements}\label{acknowledgements}}

People who have offered helpful suggestions: \href{https://github.com/pietroppeter}{@pietroppeter}

\hypertarget{prerequisites}{%
\chapter*{Prerequisites}\label{prerequisites}}
\addcontentsline{toc}{chapter}{Prerequisites}

The prerequisites in \href{https://r4ds.had.co.nz/introduction.html\#prerequisites}{R for Data Science} are the same for this guide:

\begin{enumerate}
\def\labelenumi{\arabic{enumi}.}
\tightlist
\item
  Install R for \href{https://cloud.r-project.org/bin/windows/base/}{Windows}, \href{https://cloud.r-project.org/bin/macosx/}{Mac}, or your \href{https://cloud.r-project.org/bin/linux/}{variant of Linux}.
\item
  Install \href{https://www.rstudio.com/products/rstudio/}{RStudio}.
\item
  (optional) Run RStudio, and install the tidyverse by typing the following into the RStudio console: \texttt{install.packages("tidyverse")}
\end{enumerate}

\hypertarget{variables-math-comparisons-and-strings}{%
\chapter{Variables, Math, Comparisons, and Strings}\label{variables-math-comparisons-and-strings}}

\hypertarget{help}{%
\section{Help}\label{help}}

\begin{Shaded}
\begin{Highlighting}[]
\CommentTok{\# Hi. This is a comment.}

\CommentTok{\# If you know a function\textquotesingle{}s name, but not how to use it:}
\NormalTok{?t.test}
\end{Highlighting}
\end{Shaded}

You can also mouseover a function and press F1.

If you don't know the exact name of a function or variable, you can type part of the name and press tab to autocomplete and see some info about it.

\hypertarget{assignment}{%
\section{Assignment}\label{assignment}}

\begin{Shaded}
\begin{Highlighting}[]
\NormalTok{a }\OtherTok{=} \DecValTok{6}
\NormalTok{b }\OtherTok{=} \DecValTok{8}
\NormalTok{c }\OtherTok{=} \FloatTok{5.44}
\NormalTok{d }\OtherTok{=} \ConstantTok{TRUE}
\NormalTok{e }\OtherTok{=} \StringTok{"hello world"} 
\NormalTok{e }\OtherTok{=} \StringTok{\textquotesingle{}hello world\textquotesingle{}} \CommentTok{\# same as double quote}
\end{Highlighting}
\end{Shaded}

\emph{Note: No semi colon or ``var'' needed}

You'll sometimes see \texttt{a\ \textless{}-\ 6} instead of \texttt{a\ =\ 6}. Just use \texttt{=}. Some people insist on using \texttt{\textless{}-}. They are silly.

\hypertarget{names-with-weird-characters}{%
\section{Names with weird characters}\label{names-with-weird-characters}}

R allows names to have a \texttt{.}, and it's common in many built-in functions. For your own variables, avoid it if possible. If you want to have a space in a name, use an underscore (\texttt{\_}) instead of being rediculous.

\emph{To learn how to access object members, see the} \href{lists.html\#accessing-elements-in-a-list}{lists chapter}\emph{.}

\begin{Shaded}
\begin{Highlighting}[]
\NormalTok{this.is.a.variable.name }\OtherTok{=} \DecValTok{1}
\NormalTok{better\_name }\OtherTok{=} \DecValTok{2}
\end{Highlighting}
\end{Shaded}

You can use any weird character like a space in a variable name by surrounding the name with `. Avoid it if you can, but sometimes it's necessary when you load data from a file.

\begin{Shaded}
\begin{Highlighting}[]
\StringTok{\textasciigrave{}}\AttributeTok{more than four (\textgreater{}4)}\StringTok{\textasciigrave{}} \OtherTok{=} \DecValTok{5}
\end{Highlighting}
\end{Shaded}

\hypertarget{console-output}{%
\section{Console Output}\label{console-output}}

Print \texttt{a} in the console

\begin{Shaded}
\begin{Highlighting}[]
\NormalTok{a}
\end{Highlighting}
\end{Shaded}

\begin{verbatim}
#> [1] 6
\end{verbatim}

\emph{The \texttt{{[}1{]}} is output because all values are arrays.}

Another option that's useful inside functions

\begin{Shaded}
\begin{Highlighting}[]
\FunctionTok{print}\NormalTok{(a)}
\end{Highlighting}
\end{Shaded}

\begin{verbatim}
#> [1] 6
\end{verbatim}

\hypertarget{math}{%
\section{Math}\label{math}}

Arithmetic

\begin{Shaded}
\begin{Highlighting}[]
\NormalTok{z }\OtherTok{=}\NormalTok{ a }\SpecialCharTok{+}\NormalTok{ b}
\NormalTok{z }\OtherTok{=}\NormalTok{ a }\SpecialCharTok{{-}}\NormalTok{ b}
\NormalTok{z }\OtherTok{=}\NormalTok{ a }\SpecialCharTok{*}\NormalTok{ b}
\NormalTok{z }\OtherTok{=}\NormalTok{ a }\SpecialCharTok{/}\NormalTok{ b}
\NormalTok{z }\OtherTok{=}\NormalTok{ a }\SpecialCharTok{\%/\%}\NormalTok{ b }\CommentTok{\# Integer division}
\NormalTok{z }\OtherTok{=}\NormalTok{ a }\SpecialCharTok{\%\%}\NormalTok{ b }\CommentTok{\# Note the double \% for the modulo operator}
\NormalTok{z }\OtherTok{=}\NormalTok{ a }\SpecialCharTok{\^{}}\NormalTok{ b }\CommentTok{\# exponent}

\DecValTok{1} \SpecialCharTok{+} \DecValTok{2} \SpecialCharTok{{-}} \DecValTok{3} \SpecialCharTok{*} \DecValTok{4} \SpecialCharTok{/} \DecValTok{5} \SpecialCharTok{\^{}} \DecValTok{6} \CommentTok{\# Please excuse my dear aunt, Sally}
\end{Highlighting}
\end{Shaded}

\begin{verbatim}
#> [1] 2.999232
\end{verbatim}

\emph{Note: There is no ++ or +=}

Functions for floats

\begin{Shaded}
\begin{Highlighting}[]
\FunctionTok{floor}\NormalTok{(}\FloatTok{4.82}\NormalTok{)}
\end{Highlighting}
\end{Shaded}

\begin{verbatim}
#> [1] 4
\end{verbatim}

\begin{Shaded}
\begin{Highlighting}[]
\FunctionTok{ceiling}\NormalTok{(}\FloatTok{4.82}\NormalTok{)}
\end{Highlighting}
\end{Shaded}

\begin{verbatim}
#> [1] 5
\end{verbatim}

Rounding

\begin{Shaded}
\begin{Highlighting}[]
\FunctionTok{round}\NormalTok{(}\FloatTok{4.4}\NormalTok{) }\CommentTok{\# round down}
\end{Highlighting}
\end{Shaded}

\begin{verbatim}
#> [1] 4
\end{verbatim}

\begin{Shaded}
\begin{Highlighting}[]
\FunctionTok{round}\NormalTok{(}\FloatTok{4.6}\NormalTok{) }\CommentTok{\# round up}
\end{Highlighting}
\end{Shaded}

\begin{verbatim}
#> [1] 5
\end{verbatim}

\begin{Shaded}
\begin{Highlighting}[]
\FunctionTok{round}\NormalTok{(}\FloatTok{4.5}\NormalTok{) }\CommentTok{\# round to even (down)}
\end{Highlighting}
\end{Shaded}

\begin{verbatim}
#> [1] 4
\end{verbatim}

\begin{Shaded}
\begin{Highlighting}[]
\FunctionTok{round}\NormalTok{(}\FloatTok{5.5}\NormalTok{) }\CommentTok{\# round to even (up)}
\end{Highlighting}
\end{Shaded}

\begin{verbatim}
#> [1] 6
\end{verbatim}

Other basic math functions

\begin{Shaded}
\begin{Highlighting}[]
\FunctionTok{sin}\NormalTok{(pi}\SpecialCharTok{/}\DecValTok{2}\NormalTok{) }\SpecialCharTok{+} \FunctionTok{cos}\NormalTok{(}\DecValTok{0}\NormalTok{) }\CommentTok{\# radians, not degrees}
\end{Highlighting}
\end{Shaded}

\begin{verbatim}
#> [1] 2
\end{verbatim}

\begin{Shaded}
\begin{Highlighting}[]
\FunctionTok{log}\NormalTok{(}\FunctionTok{exp}\NormalTok{(}\DecValTok{2}\NormalTok{)) }\CommentTok{\# base e (like ln) is the default}
\end{Highlighting}
\end{Shaded}

\begin{verbatim}
#> [1] 2
\end{verbatim}

\begin{Shaded}
\begin{Highlighting}[]
\FunctionTok{log}\NormalTok{(}\DecValTok{100}\NormalTok{, }\DecValTok{10}\NormalTok{) }\CommentTok{\# use base 10}
\end{Highlighting}
\end{Shaded}

\begin{verbatim}
#> [1] 2
\end{verbatim}

\hypertarget{comparisons}{%
\section{Comparisons}\label{comparisons}}

\begin{Shaded}
\begin{Highlighting}[]
\NormalTok{a }\SpecialCharTok{==}\NormalTok{ b}
\end{Highlighting}
\end{Shaded}

\begin{verbatim}
#> [1] FALSE
\end{verbatim}

\begin{Shaded}
\begin{Highlighting}[]
\NormalTok{a }\SpecialCharTok{!=}\NormalTok{ b}
\end{Highlighting}
\end{Shaded}

\begin{verbatim}
#> [1] TRUE
\end{verbatim}

\begin{Shaded}
\begin{Highlighting}[]
\NormalTok{a }\SpecialCharTok{\textgreater{}}\NormalTok{ b}
\end{Highlighting}
\end{Shaded}

\begin{verbatim}
#> [1] FALSE
\end{verbatim}

\begin{Shaded}
\begin{Highlighting}[]
\NormalTok{a }\SpecialCharTok{\textless{}}\NormalTok{ b}
\end{Highlighting}
\end{Shaded}

\begin{verbatim}
#> [1] TRUE
\end{verbatim}

\begin{Shaded}
\begin{Highlighting}[]
\NormalTok{a }\SpecialCharTok{\textgreater{}=}\NormalTok{ b}
\end{Highlighting}
\end{Shaded}

\begin{verbatim}
#> [1] FALSE
\end{verbatim}

\begin{Shaded}
\begin{Highlighting}[]
\NormalTok{a }\SpecialCharTok{\textless{}=}\NormalTok{ b}
\end{Highlighting}
\end{Shaded}

\begin{verbatim}
#> [1] TRUE
\end{verbatim}

\hypertarget{boolean}{%
\section{Boolean}\label{boolean}}

\begin{Shaded}
\begin{Highlighting}[]
\ConstantTok{TRUE} \SpecialCharTok{\&} \ConstantTok{FALSE}
\end{Highlighting}
\end{Shaded}

\begin{verbatim}
#> [1] FALSE
\end{verbatim}

\begin{Shaded}
\begin{Highlighting}[]
\ConstantTok{TRUE} \SpecialCharTok{|} \ConstantTok{FALSE}
\end{Highlighting}
\end{Shaded}

\begin{verbatim}
#> [1] TRUE
\end{verbatim}

\begin{Shaded}
\begin{Highlighting}[]
\SpecialCharTok{!}\ConstantTok{TRUE}
\end{Highlighting}
\end{Shaded}

\begin{verbatim}
#> [1] FALSE
\end{verbatim}

There's \texttt{\&} and \texttt{\&\&}. You usually want just \texttt{\&}.

\hypertarget{strings}{%
\section{Strings}\label{strings}}

Strings are not arrays in R, so array techniques may not work on strings.

String length

\begin{Shaded}
\begin{Highlighting}[]
\FunctionTok{nchar}\NormalTok{(}\StringTok{\textquotesingle{}hello world\textquotesingle{}}\NormalTok{)}
\end{Highlighting}
\end{Shaded}

\begin{verbatim}
#> [1] 11
\end{verbatim}

Substring

\begin{Shaded}
\begin{Highlighting}[]
\FunctionTok{substring}\NormalTok{(}\StringTok{\textquotesingle{}hello world\textquotesingle{}}\NormalTok{, }\DecValTok{2}\NormalTok{, }\DecValTok{10}\NormalTok{)}
\end{Highlighting}
\end{Shaded}

\begin{verbatim}
#> [1] "ello worl"
\end{verbatim}

Comparison

\begin{Shaded}
\begin{Highlighting}[]
\StringTok{\textquotesingle{}hello\textquotesingle{}} \SpecialCharTok{==} \StringTok{"hello"}
\end{Highlighting}
\end{Shaded}

\begin{verbatim}
#> [1] TRUE
\end{verbatim}

\hypertarget{strings-with-special-characters}{%
\subsection{Strings with special characters}\label{strings-with-special-characters}}

If you want to use special characters in a string, you need to ``escape it'' by adding \texttt{\textbackslash{}}

\begin{Shaded}
\begin{Highlighting}[]
\StringTok{"string with backslashes }\SpecialCharTok{\textbackslash{}\textbackslash{}}\StringTok{, double quote }\SpecialCharTok{\textbackslash{}"}\StringTok{, and unicode \textbackslash{}u263A"}
\end{Highlighting}
\end{Shaded}

\begin{verbatim}
#> [1] "string with backslashes \\, double quote \", and unicode <U+263A>"
\end{verbatim}

Or you can use the literal \texttt{r"(text)"} which is useful for a Windows path or regular expression

\begin{Shaded}
\begin{Highlighting}[]
\NormalTok{r}\StringTok{"(c:\textbackslash{}hello\textbackslash{}world)"}
\end{Highlighting}
\end{Shaded}

\begin{verbatim}
#> [1] "c:\\hello\\world"
\end{verbatim}

\hypertarget{string-concatenation}{%
\subsection{String Concatenation}\label{string-concatenation}}

Concatenate with a space in between

\begin{Shaded}
\begin{Highlighting}[]
\FunctionTok{paste}\NormalTok{(}\StringTok{\textquotesingle{}hello\textquotesingle{}}\NormalTok{, }\StringTok{\textquotesingle{}world\textquotesingle{}}\NormalTok{)}
\end{Highlighting}
\end{Shaded}

\begin{verbatim}
#> [1] "hello world"
\end{verbatim}

Use a difference separator

\begin{Shaded}
\begin{Highlighting}[]
\FunctionTok{paste}\NormalTok{(}\StringTok{\textquotesingle{}hello\textquotesingle{}}\NormalTok{, }\StringTok{\textquotesingle{}world\textquotesingle{}}\NormalTok{, }\AttributeTok{sep=}\StringTok{\textquotesingle{}\_\textquotesingle{}}\NormalTok{)}
\end{Highlighting}
\end{Shaded}

\begin{verbatim}
#> [1] "hello_world"
\end{verbatim}

No separator

\begin{Shaded}
\begin{Highlighting}[]
\FunctionTok{paste}\NormalTok{(}\StringTok{\textquotesingle{}hello\textquotesingle{}}\NormalTok{, }\StringTok{\textquotesingle{}world\textquotesingle{}}\NormalTok{, }\AttributeTok{sep=}\StringTok{\textquotesingle{}\textquotesingle{}}\NormalTok{)}
\end{Highlighting}
\end{Shaded}

\begin{verbatim}
#> [1] "helloworld"
\end{verbatim}

\hypertarget{arrays}{%
\chapter{Arrays}\label{arrays}}

In R, arrays are commonly called ``vectors''. R likes to be special.

\hypertarget{everything-is-an-array}{%
\section{Everything is an array}\label{everything-is-an-array}}

In R, even single values are arrays. That's why you see \texttt{{[}1{]}} in front of results: even single values are the first item in an array of length one.

\hypertarget{creation}{%
\section{Creation}\label{creation}}

\texttt{c()} is some sort of legacy nonsense from the S language. I think it means \emph{character array} even though it can hold things other than characters.

I pronounce it ``CAW''. Like the sound a crow makes.

Simple array

\begin{Shaded}
\begin{Highlighting}[]
\FunctionTok{c}\NormalTok{(}\DecValTok{8}\NormalTok{, }\DecValTok{6}\NormalTok{, }\DecValTok{7}\NormalTok{, }\DecValTok{5}\NormalTok{)}
\end{Highlighting}
\end{Shaded}

\begin{verbatim}
#> [1] 8 6 7 5
\end{verbatim}

For multiple types, R converts elements to the most complex type (usually a strong). For a real multi-typed collection, see \href{lists.html}{lists}

\begin{Shaded}
\begin{Highlighting}[]
\FunctionTok{c}\NormalTok{(}\DecValTok{9}\NormalTok{, }\StringTok{\textquotesingle{}hello\textquotesingle{}}\NormalTok{, }\DecValTok{7}\NormalTok{)}
\end{Highlighting}
\end{Shaded}

\begin{verbatim}
#> [1] "9"     "hello" "7"
\end{verbatim}

\hypertarget{array-generators}{%
\section{Array generators}\label{array-generators}}

R has a cultural fear of complete words. Many terms are shortcuts or acronyms.

Repeat

\begin{Shaded}
\begin{Highlighting}[]
\FunctionTok{rep}\NormalTok{(}\DecValTok{0}\NormalTok{, }\DecValTok{4}\NormalTok{)}
\end{Highlighting}
\end{Shaded}

\begin{verbatim}
#> [1] 0 0 0 0
\end{verbatim}

\begin{Shaded}
\begin{Highlighting}[]
\FunctionTok{rep}\NormalTok{(}\FunctionTok{c}\NormalTok{(}\DecValTok{1}\NormalTok{,}\DecValTok{2}\NormalTok{,}\DecValTok{3}\NormalTok{), }\DecValTok{4}\NormalTok{) }\CommentTok{\# repeate the whole array}
\end{Highlighting}
\end{Shaded}

\begin{verbatim}
#>  [1] 1 2 3 1 2 3 1 2 3 1 2 3
\end{verbatim}

\begin{Shaded}
\begin{Highlighting}[]
\FunctionTok{rep}\NormalTok{(}\FunctionTok{c}\NormalTok{(}\DecValTok{1}\NormalTok{,}\DecValTok{2}\NormalTok{,}\DecValTok{3}\NormalTok{), }\AttributeTok{each=}\DecValTok{4}\NormalTok{) }\CommentTok{\# repeat each item in the array before moving to the next}
\end{Highlighting}
\end{Shaded}

\begin{verbatim}
#>  [1] 1 1 1 1 2 2 2 2 3 3 3 3
\end{verbatim}

Sequence

\begin{Shaded}
\begin{Highlighting}[]
\CommentTok{\#increment by 1}
\DecValTok{4}\SpecialCharTok{:}\DecValTok{10}
\end{Highlighting}
\end{Shaded}

\begin{verbatim}
#> [1]  4  5  6  7  8  9 10
\end{verbatim}

\begin{Shaded}
\begin{Highlighting}[]
\CommentTok{\#increment by any other value}
\FunctionTok{seq}\NormalTok{(}\AttributeTok{from=}\DecValTok{10}\NormalTok{, }\AttributeTok{to=}\DecValTok{50}\NormalTok{, }\AttributeTok{by=}\DecValTok{5}\NormalTok{)}
\end{Highlighting}
\end{Shaded}

\begin{verbatim}
#> [1] 10 15 20 25 30 35 40 45 50
\end{verbatim}

Randomly sample from a given distribution

\begin{Shaded}
\begin{Highlighting}[]
\CommentTok{\# uniform distribution (not \textquotesingle{}run if\textquotesingle{})}
\FunctionTok{runif}\NormalTok{(}\AttributeTok{n=}\DecValTok{5}\NormalTok{, }\AttributeTok{min=}\DecValTok{0}\NormalTok{, }\AttributeTok{max=}\DecValTok{1}\NormalTok{)}
\end{Highlighting}
\end{Shaded}

\begin{verbatim}
#> [1] 0.1594836 0.4781883 0.7647987 0.7696877 0.2685485
\end{verbatim}

\begin{Shaded}
\begin{Highlighting}[]
\CommentTok{\# normal distribution}
\FunctionTok{rnorm}\NormalTok{(}\AttributeTok{n=}\DecValTok{5}\NormalTok{, }\AttributeTok{mean=}\DecValTok{0}\NormalTok{, }\AttributeTok{sd=}\DecValTok{1}\NormalTok{)}
\end{Highlighting}
\end{Shaded}

\begin{verbatim}
#> [1]  0.4483395  1.0208067 -0.1378989  0.2103863 -0.6428271
\end{verbatim}

\hypertarget{concatenation}{%
\section{Concatenation}\label{concatenation}}

An array made up of smaller arrays combines them. R doesn't seem to allow for an array of arrays.

\begin{Shaded}
\begin{Highlighting}[]
\NormalTok{x }\OtherTok{=} \DecValTok{1}\SpecialCharTok{:}\DecValTok{3}
\NormalTok{y }\OtherTok{=} \FunctionTok{c}\NormalTok{(}\DecValTok{10}\NormalTok{, }\DecValTok{11}\NormalTok{)}
\NormalTok{z }\OtherTok{=} \DecValTok{500}

\FunctionTok{c}\NormalTok{(x, y, z)}
\end{Highlighting}
\end{Shaded}

\begin{verbatim}
#> [1]   1   2   3  10  11 500
\end{verbatim}

\emph{Note: \texttt{z} is technically an array of length 1}

Collapse an array into a string

\begin{Shaded}
\begin{Highlighting}[]
\FunctionTok{paste}\NormalTok{(}\DecValTok{1}\SpecialCharTok{:}\DecValTok{5}\NormalTok{, }\AttributeTok{collapse=}\StringTok{", "}\NormalTok{)}
\end{Highlighting}
\end{Shaded}

\begin{verbatim}
#> [1] "1, 2, 3, 4, 5"
\end{verbatim}

\hypertarget{indexing}{%
\section{Indexing}\label{indexing}}

\begin{Shaded}
\begin{Highlighting}[]
\NormalTok{a }\OtherTok{=} \DecValTok{10}\SpecialCharTok{:}\DecValTok{20}
\end{Highlighting}
\end{Shaded}

Get the first value - \textbf{Indices start at 1, not 0}

\begin{Shaded}
\begin{Highlighting}[]
\NormalTok{a[}\DecValTok{1}\NormalTok{]}
\end{Highlighting}
\end{Shaded}

\begin{verbatim}
#> [1] 10
\end{verbatim}

2nd and 6th values

\begin{Shaded}
\begin{Highlighting}[]
\NormalTok{a[}\FunctionTok{c}\NormalTok{(}\DecValTok{2}\NormalTok{,}\DecValTok{6}\NormalTok{)]}
\end{Highlighting}
\end{Shaded}

\begin{verbatim}
#> [1] 11 15
\end{verbatim}

Exclude the 2nd and 6th values

\begin{Shaded}
\begin{Highlighting}[]
\NormalTok{a[}\FunctionTok{c}\NormalTok{(}\SpecialCharTok{{-}}\DecValTok{2}\NormalTok{,}\SpecialCharTok{{-}}\DecValTok{6}\NormalTok{)]}
\end{Highlighting}
\end{Shaded}

\begin{verbatim}
#> [1] 10 12 13 14 16 17 18 19 20
\end{verbatim}

Range of values

\begin{Shaded}
\begin{Highlighting}[]
\NormalTok{a[}\DecValTok{2}\SpecialCharTok{:}\DecValTok{6}\NormalTok{]}
\end{Highlighting}
\end{Shaded}

\begin{verbatim}
#> [1] 11 12 13 14 15
\end{verbatim}

Any order or number of repetitions

\begin{Shaded}
\begin{Highlighting}[]
\NormalTok{a[}\FunctionTok{c}\NormalTok{(}\DecValTok{2}\NormalTok{, }\DecValTok{4}\NormalTok{, }\DecValTok{6}\NormalTok{, }\DecValTok{6}\NormalTok{, }\DecValTok{6}\NormalTok{)]}
\end{Highlighting}
\end{Shaded}

\begin{verbatim}
#> [1] 11 13 15 15 15
\end{verbatim}

specify values using booleans (keep this in mind for the ``Array operators'' section)

\begin{Shaded}
\begin{Highlighting}[]
\NormalTok{a[}\FunctionTok{c}\NormalTok{(}\ConstantTok{TRUE}\NormalTok{, }\ConstantTok{FALSE}\NormalTok{, }\ConstantTok{TRUE}\NormalTok{, }\ConstantTok{FALSE}\NormalTok{, }\ConstantTok{TRUE}\NormalTok{, }\ConstantTok{FALSE}\NormalTok{, }\ConstantTok{TRUE}\NormalTok{, }\ConstantTok{FALSE}\NormalTok{, }\ConstantTok{TRUE}\NormalTok{, }\ConstantTok{FALSE}\NormalTok{)]}
\end{Highlighting}
\end{Shaded}

\begin{verbatim}
#> [1] 10 12 14 16 18 20
\end{verbatim}

\hypertarget{sampling-from-an-array}{%
\section{Sampling from an Array}\label{sampling-from-an-array}}

Randomly sample from an array. Elements may repeat.

\begin{Shaded}
\begin{Highlighting}[]
\FunctionTok{sample}\NormalTok{(}\DecValTok{1}\SpecialCharTok{:}\DecValTok{3}\NormalTok{, }\AttributeTok{size=}\DecValTok{10}\NormalTok{, }\AttributeTok{replace=}\ConstantTok{TRUE}\NormalTok{)}
\end{Highlighting}
\end{Shaded}

\begin{verbatim}
#>  [1] 1 1 2 3 2 2 2 2 3 1
\end{verbatim}

\emph{\texttt{replace} means ``sample with replacement'', so an element can be sampled more than once}

Sample without replacement. Elements will not repeat.

\begin{Shaded}
\begin{Highlighting}[]
\FunctionTok{sample}\NormalTok{(}\DecValTok{1}\SpecialCharTok{:}\DecValTok{5}\NormalTok{, }\AttributeTok{size=}\DecValTok{4}\NormalTok{, }\AttributeTok{replace=}\ConstantTok{FALSE}\NormalTok{)}
\end{Highlighting}
\end{Shaded}

\begin{verbatim}
#> [1] 4 3 1 5
\end{verbatim}

Shuffle the order of an array

\begin{Shaded}
\begin{Highlighting}[]
\FunctionTok{sample}\NormalTok{(a, }\AttributeTok{size=}\FunctionTok{length}\NormalTok{(a), }\AttributeTok{replace=}\ConstantTok{FALSE}\NormalTok{)}
\end{Highlighting}
\end{Shaded}

\begin{verbatim}
#>  [1] 15 13 17 20 18 10 12 16 14 11 19
\end{verbatim}

Make sure you have enough elements

\begin{Shaded}
\begin{Highlighting}[]
\FunctionTok{sample}\NormalTok{(}\DecValTok{1}\SpecialCharTok{:}\DecValTok{5}\NormalTok{, }\AttributeTok{size=}\DecValTok{10}\NormalTok{, }\AttributeTok{replace=}\ConstantTok{FALSE}\NormalTok{)}
\end{Highlighting}
\end{Shaded}

\begin{verbatim}
#> Error in sample.int(length(x), size, replace, prob): cannot take a sample larger than the population when 'replace = FALSE'
\end{verbatim}

\hypertarget{array-constants}{%
\section{Array constants}\label{array-constants}}

The \texttt{letters} and \texttt{LETTERS} constants hold lower and upper case letters

\begin{Shaded}
\begin{Highlighting}[]
\NormalTok{letters[}\DecValTok{1}\SpecialCharTok{:}\DecValTok{5}\NormalTok{]}
\end{Highlighting}
\end{Shaded}

\begin{verbatim}
#> [1] "a" "b" "c" "d" "e"
\end{verbatim}

\begin{Shaded}
\begin{Highlighting}[]
\NormalTok{LETTERS[}\DecValTok{1}\SpecialCharTok{:}\DecValTok{5}\NormalTok{]}
\end{Highlighting}
\end{Shaded}

\begin{verbatim}
#> [1] "A" "B" "C" "D" "E"
\end{verbatim}

\hypertarget{array-operations}{%
\section{Array operations}\label{array-operations}}

Compare individual elements

\begin{Shaded}
\begin{Highlighting}[]
\NormalTok{a }\SpecialCharTok{\textgreater{}} \DecValTok{15}
\end{Highlighting}
\end{Shaded}

\begin{verbatim}
#>  [1] FALSE FALSE FALSE FALSE FALSE FALSE  TRUE  TRUE  TRUE  TRUE  TRUE
\end{verbatim}

Compare each element across arrays

\begin{Shaded}
\begin{Highlighting}[]
\NormalTok{a }\SpecialCharTok{==} \FunctionTok{c}\NormalTok{(}\DecValTok{10}\NormalTok{, }\DecValTok{9}\NormalTok{, }\DecValTok{12}\NormalTok{, }\DecValTok{13}\NormalTok{, }\DecValTok{14}\NormalTok{, }\DecValTok{15}\NormalTok{, }\DecValTok{16}\NormalTok{, }\DecValTok{17}\NormalTok{, }\DecValTok{18}\NormalTok{, }\DecValTok{19}\NormalTok{, }\DecValTok{20}\NormalTok{)}
\end{Highlighting}
\end{Shaded}

\begin{verbatim}
#>  [1]  TRUE FALSE  TRUE  TRUE  TRUE  TRUE  TRUE  TRUE  TRUE  TRUE  TRUE
\end{verbatim}

Select elements using boolean array

\begin{Shaded}
\begin{Highlighting}[]
\NormalTok{a[a}\SpecialCharTok{\textgreater{}}\DecValTok{15}\NormalTok{]}
\end{Highlighting}
\end{Shaded}

\begin{verbatim}
#> [1] 16 17 18 19 20
\end{verbatim}

You can perform operations on the elements of two arrays \textbf{even if they are different sizes}. The smaller one wraps around.

\begin{Shaded}
\begin{Highlighting}[]
\NormalTok{a }\OtherTok{=} \DecValTok{1}\SpecialCharTok{:}\DecValTok{5}
\NormalTok{b }\OtherTok{=} \FunctionTok{rep}\NormalTok{(}\DecValTok{1}\NormalTok{, }\DecValTok{8}\NormalTok{)}
\NormalTok{a }\SpecialCharTok{+}\NormalTok{ b}
\end{Highlighting}
\end{Shaded}

\begin{verbatim}
#> Warning in a + b: longer object length is not a multiple of shorter object
#> length
\end{verbatim}

\begin{verbatim}
#> [1] 2 3 4 5 6 2 3 4
\end{verbatim}

Because all variables are arrays, scalars work the same way:

\begin{Shaded}
\begin{Highlighting}[]
\NormalTok{a }\SpecialCharTok{+} \DecValTok{1}
\end{Highlighting}
\end{Shaded}

\begin{verbatim}
#> [1] 2 3 4 5 6
\end{verbatim}

\hypertarget{array-functions}{%
\section{Array functions}\label{array-functions}}

Length

\begin{Shaded}
\begin{Highlighting}[]
\FunctionTok{length}\NormalTok{(}\DecValTok{20}\SpecialCharTok{:}\DecValTok{50}\NormalTok{)}
\end{Highlighting}
\end{Shaded}

\begin{verbatim}
#> [1] 31
\end{verbatim}

Reverse

\begin{Shaded}
\begin{Highlighting}[]
\FunctionTok{rev}\NormalTok{(}\DecValTok{1}\SpecialCharTok{:}\DecValTok{5}\NormalTok{)}
\end{Highlighting}
\end{Shaded}

\begin{verbatim}
#> [1] 5 4 3 2 1
\end{verbatim}

Math

\begin{Shaded}
\begin{Highlighting}[]
\FunctionTok{min}\NormalTok{(}\DecValTok{1}\SpecialCharTok{:}\DecValTok{5}\NormalTok{)}
\end{Highlighting}
\end{Shaded}

\begin{verbatim}
#> [1] 1
\end{verbatim}

\begin{Shaded}
\begin{Highlighting}[]
\FunctionTok{max}\NormalTok{(}\DecValTok{1}\SpecialCharTok{:}\DecValTok{5}\NormalTok{)}
\end{Highlighting}
\end{Shaded}

\begin{verbatim}
#> [1] 5
\end{verbatim}

\begin{Shaded}
\begin{Highlighting}[]
\FunctionTok{sum}\NormalTok{(}\DecValTok{1}\SpecialCharTok{:}\DecValTok{5}\NormalTok{)}
\end{Highlighting}
\end{Shaded}

\begin{verbatim}
#> [1] 15
\end{verbatim}

\hypertarget{array-sorting}{%
\section{Array sorting}\label{array-sorting}}

Sort

\begin{Shaded}
\begin{Highlighting}[]
\NormalTok{a }\OtherTok{=} \FunctionTok{c}\NormalTok{(}\DecValTok{70}\NormalTok{, }\DecValTok{20}\NormalTok{, }\DecValTok{80}\NormalTok{, }\DecValTok{20}\NormalTok{, }\DecValTok{10}\NormalTok{, }\DecValTok{40}\NormalTok{)}
\FunctionTok{sort}\NormalTok{(a)}
\end{Highlighting}
\end{Shaded}

\begin{verbatim}
#> [1] 10 20 20 40 70 80
\end{verbatim}

Reverse

\begin{Shaded}
\begin{Highlighting}[]
\FunctionTok{sort}\NormalTok{(a, }\AttributeTok{decreasing=}\ConstantTok{TRUE}\NormalTok{)}
\end{Highlighting}
\end{Shaded}

\begin{verbatim}
#> [1] 80 70 40 20 20 10
\end{verbatim}

Get the indices of the sorted values

\begin{Shaded}
\begin{Highlighting}[]
\FunctionTok{order}\NormalTok{(a)}
\end{Highlighting}
\end{Shaded}

\begin{verbatim}
#> [1] 5 2 4 6 1 3
\end{verbatim}

\hypertarget{test-membership}{%
\section{Test membership}\label{test-membership}}

To see if an item is in an array, use \texttt{\%in\%}

\begin{Shaded}
\begin{Highlighting}[]
\DecValTok{9} \SpecialCharTok{\%in\%} \DecValTok{1}\SpecialCharTok{:}\DecValTok{10}
\end{Highlighting}
\end{Shaded}

\begin{verbatim}
#> [1] TRUE
\end{verbatim}

\begin{Shaded}
\begin{Highlighting}[]
\DecValTok{9}\SpecialCharTok{:}\DecValTok{11} \SpecialCharTok{\%in\%} \DecValTok{1}\SpecialCharTok{:}\DecValTok{10}
\end{Highlighting}
\end{Shaded}

\begin{verbatim}
#> [1]  TRUE  TRUE FALSE
\end{verbatim}

\hypertarget{types}{%
\chapter{Types}\label{types}}

\hypertarget{numbers}{%
\section{Numbers}\label{numbers}}

R has integers but defaults all numbers to \texttt{numeric} which is a double precision float

\begin{Shaded}
\begin{Highlighting}[]
\NormalTok{x }\OtherTok{=} \DecValTok{5} \CommentTok{\# no decimal but still a double}
\NormalTok{y }\OtherTok{=}\NormalTok{ x }\SpecialCharTok{+} \DecValTok{1}
\end{Highlighting}
\end{Shaded}

Good ol' float point comparison

\begin{Shaded}
\begin{Highlighting}[]
\NormalTok{x }\OtherTok{=}\NormalTok{ .}\DecValTok{58}
\NormalTok{y }\OtherTok{=}\NormalTok{ .}\DecValTok{08}
\NormalTok{x }\SpecialCharTok{{-}}\NormalTok{ y }\SpecialCharTok{==} \FloatTok{0.5}
\end{Highlighting}
\end{Shaded}

\begin{verbatim}
#> [1] FALSE
\end{verbatim}

\begin{Shaded}
\begin{Highlighting}[]
\FunctionTok{round}\NormalTok{(x}\SpecialCharTok{{-}}\NormalTok{y, }\AttributeTok{digits=}\DecValTok{1}\NormalTok{) }\SpecialCharTok{==} \FunctionTok{round}\NormalTok{(}\FloatTok{0.5}\NormalTok{, }\AttributeTok{digits=}\DecValTok{1}\NormalTok{)}
\end{Highlighting}
\end{Shaded}

\begin{verbatim}
#> [1] TRUE
\end{verbatim}

Numeric division returns double

\begin{Shaded}
\begin{Highlighting}[]
\DecValTok{9} \SpecialCharTok{/} \DecValTok{2} \CommentTok{\# double}
\end{Highlighting}
\end{Shaded}

\begin{verbatim}
#> [1] 4.5
\end{verbatim}

\begin{Shaded}
\begin{Highlighting}[]
\DecValTok{9} \SpecialCharTok{\%/\%} \DecValTok{2} \CommentTok{\# drop the part after the decimal}
\end{Highlighting}
\end{Shaded}

\begin{verbatim}
#> [1] 4
\end{verbatim}

\hypertarget{strings-1}{%
\section{Strings}\label{strings-1}}

Single and double quotes are the same in R, but a given string needs the same in the beginning and end

\begin{Shaded}
\begin{Highlighting}[]
\StringTok{"hello world"}
\end{Highlighting}
\end{Shaded}

\begin{verbatim}
#> [1] "hello world"
\end{verbatim}

\begin{Shaded}
\begin{Highlighting}[]
\StringTok{\textquotesingle{}hello world\textquotesingle{}}
\end{Highlighting}
\end{Shaded}

\begin{verbatim}
#> [1] "hello world"
\end{verbatim}

\begin{Shaded}
\begin{Highlighting}[]
\StringTok{"single quote \textquotesingle{} in a string"}
\end{Highlighting}
\end{Shaded}

\begin{verbatim}
#> [1] "single quote ' in a string"
\end{verbatim}

\begin{Shaded}
\begin{Highlighting}[]
\StringTok{\textquotesingle{}double quote " in a string\textquotesingle{}}
\end{Highlighting}
\end{Shaded}

\begin{verbatim}
#> [1] "double quote \" in a string"
\end{verbatim}

\hypertarget{concatenation-1}{%
\subsection{Concatenation}\label{concatenation-1}}

Concatenate with a space in between

\begin{Shaded}
\begin{Highlighting}[]
\FunctionTok{paste}\NormalTok{(}\StringTok{\textquotesingle{}hello\textquotesingle{}}\NormalTok{, }\StringTok{\textquotesingle{}world\textquotesingle{}}\NormalTok{)}
\end{Highlighting}
\end{Shaded}

\begin{verbatim}
#> [1] "hello world"
\end{verbatim}

Use a difference separator

\begin{Shaded}
\begin{Highlighting}[]
\FunctionTok{paste}\NormalTok{(}\StringTok{\textquotesingle{}hello\textquotesingle{}}\NormalTok{, }\StringTok{\textquotesingle{}world\textquotesingle{}}\NormalTok{, }\AttributeTok{sep=}\StringTok{\textquotesingle{}\_\textquotesingle{}}\NormalTok{)}
\end{Highlighting}
\end{Shaded}

\begin{verbatim}
#> [1] "hello_world"
\end{verbatim}

No separator

\begin{Shaded}
\begin{Highlighting}[]
\FunctionTok{paste0}\NormalTok{(}\StringTok{\textquotesingle{}hello\textquotesingle{}}\NormalTok{, }\StringTok{\textquotesingle{}world\textquotesingle{}}\NormalTok{)}
\end{Highlighting}
\end{Shaded}

\begin{verbatim}
#> [1] "helloworld"
\end{verbatim}

\hypertarget{dates}{%
\section{Dates}\label{dates}}

See the \href{https://lubridate.tidyverse.org/}{lubridate library}.

\hypertarget{finding-the-type-of-a-variable}{%
\section{Finding the type of a variable}\label{finding-the-type-of-a-variable}}

\begin{Shaded}
\begin{Highlighting}[]
\FunctionTok{class}\NormalTok{(}\FunctionTok{c}\NormalTok{(}\DecValTok{5}\NormalTok{, }\StringTok{\textquotesingle{}hi\textquotesingle{}}\NormalTok{, }\ConstantTok{TRUE}\NormalTok{))}
\end{Highlighting}
\end{Shaded}

\begin{verbatim}
#> [1] "character"
\end{verbatim}

\hypertarget{checking-the-type}{%
\section{Checking the type}\label{checking-the-type}}

What's the type?

\begin{Shaded}
\begin{Highlighting}[]
\FunctionTok{class}\NormalTok{(}\DecValTok{5}\NormalTok{)}
\end{Highlighting}
\end{Shaded}

\begin{verbatim}
#> [1] "numeric"
\end{verbatim}

Remember, arrays are the same as single values.

\begin{Shaded}
\begin{Highlighting}[]
\FunctionTok{class}\NormalTok{(}\DecValTok{1}\SpecialCharTok{:}\DecValTok{5}\NormalTok{)}
\end{Highlighting}
\end{Shaded}

\begin{verbatim}
#> [1] "integer"
\end{verbatim}

Test if numeric

\begin{Shaded}
\begin{Highlighting}[]
\FunctionTok{is.numeric}\NormalTok{(}\DecValTok{5}\NormalTok{)}
\end{Highlighting}
\end{Shaded}

\begin{verbatim}
#> [1] TRUE
\end{verbatim}

Test if string

\begin{Shaded}
\begin{Highlighting}[]
\FunctionTok{is.character}\NormalTok{(}\StringTok{\textquotesingle{}hi\textquotesingle{}}\NormalTok{)}
\end{Highlighting}
\end{Shaded}

\begin{verbatim}
#> [1] TRUE
\end{verbatim}

Test if boolean

\begin{Shaded}
\begin{Highlighting}[]
\FunctionTok{is.logical}\NormalTok{(}\ConstantTok{TRUE}\NormalTok{)}
\end{Highlighting}
\end{Shaded}

\begin{verbatim}
#> [1] TRUE
\end{verbatim}

\hypertarget{converting-and-parsing}{%
\section{Converting and parsing}\label{converting-and-parsing}}

Parse or convert to numeric

\begin{Shaded}
\begin{Highlighting}[]
\FunctionTok{as.numeric}\NormalTok{(}\FunctionTok{c}\NormalTok{(}\StringTok{"5"}\NormalTok{, }\ConstantTok{TRUE}\NormalTok{, }\DecValTok{1}\SpecialCharTok{:}\DecValTok{3}\NormalTok{, }\StringTok{"abc"}\NormalTok{))}
\end{Highlighting}
\end{Shaded}

\begin{verbatim}
#> Warning: NAs introduced by coercion
\end{verbatim}

\begin{verbatim}
#> [1]  5 NA  1  2  3 NA
\end{verbatim}

To string

\begin{Shaded}
\begin{Highlighting}[]
\FunctionTok{as.character}\NormalTok{(}\DecValTok{5}\NormalTok{)}
\end{Highlighting}
\end{Shaded}

\begin{verbatim}
#> [1] "5"
\end{verbatim}

\begin{Shaded}
\begin{Highlighting}[]
\FunctionTok{format}\NormalTok{(}\DecValTok{1}\SpecialCharTok{/}\DecValTok{3}\NormalTok{)}
\end{Highlighting}
\end{Shaded}

\begin{verbatim}
#> [1] "0.3333333"
\end{verbatim}

\begin{Shaded}
\begin{Highlighting}[]
\FunctionTok{format}\NormalTok{(}\DecValTok{1}\SpecialCharTok{/}\DecValTok{3}\NormalTok{ , }\AttributeTok{digits =} \DecValTok{16}\NormalTok{)}
\end{Highlighting}
\end{Shaded}

\begin{verbatim}
#> [1] "0.3333333333333333"
\end{verbatim}

\begin{Shaded}
\begin{Highlighting}[]
\FunctionTok{as.character}\NormalTok{(}\ConstantTok{TRUE}\NormalTok{)}
\end{Highlighting}
\end{Shaded}

\begin{verbatim}
#> [1] "TRUE"
\end{verbatim}

Convert to boolean. Zero is false. Other numbers are true.

\begin{Shaded}
\begin{Highlighting}[]
\FunctionTok{as.logical}\NormalTok{(}\DecValTok{0}\SpecialCharTok{:}\DecValTok{2}\NormalTok{)}
\end{Highlighting}
\end{Shaded}

\begin{verbatim}
#> [1] FALSE  TRUE  TRUE
\end{verbatim}

\hypertarget{special-types}{%
\section{Special types}\label{special-types}}

\hypertarget{na}{%
\subsection{NA}\label{na}}

Missing values are very common in datasets.

\begin{Shaded}
\begin{Highlighting}[]
\FunctionTok{is.na}\NormalTok{(}\FunctionTok{c}\NormalTok{(}\ConstantTok{NA}\NormalTok{, }\DecValTok{1}\NormalTok{, }\StringTok{""}\NormalTok{))}
\end{Highlighting}
\end{Shaded}

\begin{verbatim}
#> [1]  TRUE FALSE FALSE
\end{verbatim}

Any operation performed on NA will also yield NA. So, you can operate on arrays with missing values.

\begin{Shaded}
\begin{Highlighting}[]
\FunctionTok{c}\NormalTok{(}\DecValTok{5}\NormalTok{, }\ConstantTok{NA}\NormalTok{, }\DecValTok{7}\NormalTok{) }\SpecialCharTok{+} \DecValTok{1}
\end{Highlighting}
\end{Shaded}

\begin{verbatim}
#> [1]  6 NA  8
\end{verbatim}

Be careful about aggregation functions like \texttt{min()}, \texttt{max()}, and \texttt{mean()}. To ignore NAs, use the \texttt{na.rm} parameter.

\begin{Shaded}
\begin{Highlighting}[]
\FunctionTok{mean}\NormalTok{(}\FunctionTok{c}\NormalTok{(}\DecValTok{5}\NormalTok{, }\ConstantTok{NA}\NormalTok{, }\DecValTok{7}\NormalTok{), }\AttributeTok{na.rm=}\ConstantTok{TRUE}\NormalTok{)}
\end{Highlighting}
\end{Shaded}

\begin{verbatim}
#> [1] 6
\end{verbatim}

\hypertarget{factor}{%
\subsection{Factor}\label{factor}}

A factor is like an enum in other languages. It encodes strings as integers via a dictionary.

Create an array with many repeating values

\begin{Shaded}
\begin{Highlighting}[]
\NormalTok{data }\OtherTok{=} \FunctionTok{sample}\NormalTok{(}\FunctionTok{c}\NormalTok{(}\StringTok{"hello"}\NormalTok{, }\StringTok{"cruel"}\NormalTok{, }\StringTok{"world"}\NormalTok{), }\DecValTok{12}\NormalTok{, }\AttributeTok{replace=}\ConstantTok{TRUE}\NormalTok{)}
\NormalTok{data}
\end{Highlighting}
\end{Shaded}

\begin{verbatim}
#>  [1] "world" "cruel" "world" "hello" "world" "hello" "cruel" "world" "world"
#> [10] "hello" "cruel" "world"
\end{verbatim}

Make it into a \texttt{factor}

\begin{Shaded}
\begin{Highlighting}[]
\NormalTok{data }\OtherTok{=} \FunctionTok{factor}\NormalTok{(data)}
\NormalTok{data}
\end{Highlighting}
\end{Shaded}

\begin{verbatim}
#>  [1] world cruel world hello world hello cruel world world hello cruel world
#> Levels: cruel hello world
\end{verbatim}

\emph{Note: The values are in the order they appear in the array}

The array is now an integer array with a dictionary

\begin{Shaded}
\begin{Highlighting}[]
\FunctionTok{as.numeric}\NormalTok{(data)}
\end{Highlighting}
\end{Shaded}

\begin{verbatim}
#>  [1] 3 1 3 2 3 2 1 3 3 2 1 3
\end{verbatim}

\begin{Shaded}
\begin{Highlighting}[]
\NormalTok{data[}\DecValTok{5}\NormalTok{]}
\end{Highlighting}
\end{Shaded}

\begin{verbatim}
#> [1] world
#> Levels: cruel hello world
\end{verbatim}

See the different values in the array

\begin{Shaded}
\begin{Highlighting}[]
\FunctionTok{levels}\NormalTok{(data)}
\end{Highlighting}
\end{Shaded}

\begin{verbatim}
#> [1] "cruel" "hello" "world"
\end{verbatim}

For more info, see \href{https://forcats.tidyverse.org/}{forcats}.

\hypertarget{control-flow}{%
\chapter{Control Flow}\label{control-flow}}

R is primarily a functional language, so you often don't need control flow yourself. But if you want to, go for it. If you can write some quick code with a for loop, go for it! Tell the R bullies to fuck off. Do what feels comfortable to you.

\hypertarget{if}{%
\section{If}\label{if}}

Simple if

\begin{Shaded}
\begin{Highlighting}[]
\NormalTok{a }\OtherTok{=} \ConstantTok{TRUE}
\ControlFlowTok{if}\NormalTok{ (a)  }
  \FunctionTok{print}\NormalTok{(}\StringTok{"a is TRUE"}\NormalTok{)}
\end{Highlighting}
\end{Shaded}

\begin{verbatim}
#> [1] "a is TRUE"
\end{verbatim}

\begin{Shaded}
\begin{Highlighting}[]
\CommentTok{\# conditionally run multiple expressions}
\ControlFlowTok{if}\NormalTok{ (a) \{}
  \FunctionTok{print}\NormalTok{(}\StringTok{"a is TRUE"}\NormalTok{)}
  \FunctionTok{print}\NormalTok{(}\StringTok{"a is TRUE"}\NormalTok{)}
\NormalTok{\}}
\end{Highlighting}
\end{Shaded}

\begin{verbatim}
#> [1] "a is TRUE"
#> [1] "a is TRUE"
\end{verbatim}

If Else

\begin{Shaded}
\begin{Highlighting}[]
\NormalTok{x }\OtherTok{=} \DecValTok{5}
\NormalTok{y }\OtherTok{=} \DecValTok{8}
\ControlFlowTok{if}\NormalTok{ (x }\SpecialCharTok{\textgreater{}}\NormalTok{ y) \{}
  \FunctionTok{print}\NormalTok{(}\StringTok{"x is greater than y"}\NormalTok{)}
\NormalTok{\} }\ControlFlowTok{else}\NormalTok{ \{}
  \FunctionTok{print}\NormalTok{(}\StringTok{"x is less than or equal to y"}\NormalTok{)}
\NormalTok{\}}
\end{Highlighting}
\end{Shaded}

\begin{verbatim}
#> [1] "x is less than or equal to y"
\end{verbatim}

The \texttt{ifelse} function is the way to handle vector operations. It is like a vectorized version of \texttt{?\ :} in C or javascript.

\begin{Shaded}
\begin{Highlighting}[]
\NormalTok{x }\OtherTok{=} \DecValTok{1}\SpecialCharTok{:}\DecValTok{10}
\FunctionTok{ifelse}\NormalTok{(x }\SpecialCharTok{\%\%} \DecValTok{2} \SpecialCharTok{==} \DecValTok{0}\NormalTok{, }\StringTok{"even"}\NormalTok{, }\StringTok{"odd"}\NormalTok{)}
\end{Highlighting}
\end{Shaded}

\begin{verbatim}
#>  [1] "odd"  "even" "odd"  "even" "odd"  "even" "odd"  "even" "odd"  "even"
\end{verbatim}

\hypertarget{while}{%
\section{While}\label{while}}

\begin{Shaded}
\begin{Highlighting}[]
\NormalTok{x }\OtherTok{=} \FunctionTok{runif}\NormalTok{(}\DecValTok{1}\NormalTok{)}
\ControlFlowTok{while}\NormalTok{ (x }\SpecialCharTok{\textless{}} \FloatTok{0.95}\NormalTok{) \{}
\NormalTok{  x }\OtherTok{=} \FunctionTok{runif}\NormalTok{(}\DecValTok{1}\NormalTok{)}
\NormalTok{\}}
\end{Highlighting}
\end{Shaded}

\hypertarget{for}{%
\section{For}\label{for}}

For works like foreach in other languages.

\begin{Shaded}
\begin{Highlighting}[]
\NormalTok{a }\OtherTok{=} \FunctionTok{runif}\NormalTok{(}\DecValTok{100}\NormalTok{, }\DecValTok{1}\NormalTok{, }\DecValTok{100}\NormalTok{)}
\ControlFlowTok{for}\NormalTok{ (x }\ControlFlowTok{in}\NormalTok{ a) \{}
  \ControlFlowTok{if}\NormalTok{ (x }\SpecialCharTok{\textgreater{}} \DecValTok{95}\NormalTok{)}
    \FunctionTok{print}\NormalTok{(x)}
\NormalTok{\}}
\end{Highlighting}
\end{Shaded}

\begin{verbatim}
#> [1] 97.59641
#> [1] 97.67089
#> [1] 95.54705
\end{verbatim}

\hypertarget{functions}{%
\chapter{Functions}\label{functions}}

Basic function

\begin{Shaded}
\begin{Highlighting}[]
\NormalTok{foo }\OtherTok{=} \ControlFlowTok{function}\NormalTok{ () \{}
  \FunctionTok{print}\NormalTok{(}\StringTok{"hello world"}\NormalTok{)}
\NormalTok{\}}
\FunctionTok{foo}\NormalTok{()}
\end{Highlighting}
\end{Shaded}

\begin{verbatim}
#> [1] "hello world"
\end{verbatim}

\emph{Note: in the function, you need to use \texttt{print} to output}

\hypertarget{parameters}{%
\section{Parameters}\label{parameters}}

Parameters and return values

\begin{Shaded}
\begin{Highlighting}[]
\NormalTok{addOne }\OtherTok{=} \ControlFlowTok{function}\NormalTok{ (x) \{}
  \FunctionTok{return}\NormalTok{(x }\SpecialCharTok{+} \DecValTok{1}\NormalTok{)}
\NormalTok{\}}
\FunctionTok{addOne}\NormalTok{(}\DecValTok{5}\NormalTok{)}
\end{Highlighting}
\end{Shaded}

\begin{verbatim}
#> [1] 6
\end{verbatim}

\emph{The syntax for return is like a function: \texttt{return(value)}}

Parameter order can be arbitrary

\begin{Shaded}
\begin{Highlighting}[]
\NormalTok{add }\OtherTok{=} \ControlFlowTok{function}\NormalTok{ (x, y) \{}
  \FunctionTok{return}\NormalTok{(x }\SpecialCharTok{+}\NormalTok{ (y}\SpecialCharTok{*}\DecValTok{10}\NormalTok{))}
\NormalTok{\}}
\FunctionTok{add}\NormalTok{(}\AttributeTok{x=}\DecValTok{2}\NormalTok{, }\AttributeTok{y=}\DecValTok{10}\NormalTok{)}
\end{Highlighting}
\end{Shaded}

\begin{verbatim}
#> [1] 102
\end{verbatim}

\begin{Shaded}
\begin{Highlighting}[]
\FunctionTok{add}\NormalTok{(}\AttributeTok{y=}\DecValTok{10}\NormalTok{, }\AttributeTok{x=}\DecValTok{2}\NormalTok{)}
\end{Highlighting}
\end{Shaded}

\begin{verbatim}
#> [1] 102
\end{verbatim}

Functions are vectorized by default

\begin{Shaded}
\begin{Highlighting}[]
\FunctionTok{addOne}\NormalTok{(}\DecValTok{1}\SpecialCharTok{:}\DecValTok{5}\NormalTok{)}
\end{Highlighting}
\end{Shaded}

\begin{verbatim}
#> [1] 2 3 4 5 6
\end{verbatim}

All parameters are pass-by-value because functions are immutable.

\begin{Shaded}
\begin{Highlighting}[]
\NormalTok{a }\OtherTok{=} \DecValTok{5}
\NormalTok{foo }\OtherTok{=} \ControlFlowTok{function}\NormalTok{ (a) \{}
\NormalTok{  a }\OtherTok{=} \DecValTok{6}
  \FunctionTok{print}\NormalTok{(}\FunctionTok{paste}\NormalTok{(}\StringTok{"Inside the function as a parameter: "}\NormalTok{, a))}
\NormalTok{\}}
\FunctionTok{print}\NormalTok{(}\FunctionTok{paste}\NormalTok{(}\StringTok{"Before the function: "}\NormalTok{, a))}
\end{Highlighting}
\end{Shaded}

\begin{verbatim}
#> [1] "Before the function:  5"
\end{verbatim}

\begin{Shaded}
\begin{Highlighting}[]
\FunctionTok{foo}\NormalTok{(}\DecValTok{1}\NormalTok{)}
\end{Highlighting}
\end{Shaded}

\begin{verbatim}
#> [1] "Inside the function as a parameter:  6"
\end{verbatim}

\begin{Shaded}
\begin{Highlighting}[]
\FunctionTok{print}\NormalTok{(}\FunctionTok{paste}\NormalTok{(}\StringTok{"After the function: "}\NormalTok{, a))}
\end{Highlighting}
\end{Shaded}

\begin{verbatim}
#> [1] "After the function:  5"
\end{verbatim}

\hypertarget{scope}{%
\section{Scope}\label{scope}}

When you assign a value inside a function, it creates a local variable in the scope of the function. You can't access the global variable inside the function. (OK, you can, but the syntax is so obnoxious that I pretend it doesn't exist)

\begin{Shaded}
\begin{Highlighting}[]
\NormalTok{a }\OtherTok{=} \DecValTok{5}
\NormalTok{foo }\OtherTok{=} \ControlFlowTok{function}\NormalTok{ () \{}
\NormalTok{  a }\OtherTok{=} \DecValTok{6}
\NormalTok{  b }\OtherTok{=} \DecValTok{100}
  \FunctionTok{print}\NormalTok{(}\FunctionTok{paste}\NormalTok{(}\StringTok{"Inside the function a ="}\NormalTok{, a))}
  \FunctionTok{print}\NormalTok{(}\FunctionTok{paste}\NormalTok{(}\StringTok{"Inside the function b ="}\NormalTok{, b))}
\NormalTok{\}}
\FunctionTok{print}\NormalTok{(}\FunctionTok{paste}\NormalTok{(}\StringTok{"Before the function a ="}\NormalTok{, a))}
\end{Highlighting}
\end{Shaded}

\begin{verbatim}
#> [1] "Before the function a = 5"
\end{verbatim}

\begin{Shaded}
\begin{Highlighting}[]
\FunctionTok{foo}\NormalTok{()}
\end{Highlighting}
\end{Shaded}

\begin{verbatim}
#> [1] "Inside the function a = 6"
#> [1] "Inside the function b = 100"
\end{verbatim}

\begin{Shaded}
\begin{Highlighting}[]
\FunctionTok{print}\NormalTok{(}\FunctionTok{paste}\NormalTok{(}\StringTok{"After the function a ="}\NormalTok{, a))}
\end{Highlighting}
\end{Shaded}

\begin{verbatim}
#> [1] "After the function a = 5"
\end{verbatim}

\begin{Shaded}
\begin{Highlighting}[]
\CommentTok{\#trying to use \textasciigrave{}b\textasciigrave{} will cause an error because it is out of scope}
\end{Highlighting}
\end{Shaded}

\hypertarget{a-function-in-a-function}{%
\section{A function in a function}\label{a-function-in-a-function}}

Might be useful for encapsulation

\begin{Shaded}
\begin{Highlighting}[]
\NormalTok{foo }\OtherTok{=} \ControlFlowTok{function}\NormalTok{ (a, b) \{}
\NormalTok{  square }\OtherTok{=} \ControlFlowTok{function}\NormalTok{(x) \{}
    \FunctionTok{return}\NormalTok{(x }\SpecialCharTok{\^{}}\NormalTok{ x)}
\NormalTok{  \}}
  \FunctionTok{print}\NormalTok{(}\FunctionTok{c}\NormalTok{(a, b))}
  \FunctionTok{print}\NormalTok{(}\FunctionTok{c}\NormalTok{(}\FunctionTok{square}\NormalTok{(a), }\FunctionTok{square}\NormalTok{(b)))}
\NormalTok{\}}
\FunctionTok{foo}\NormalTok{(}\DecValTok{1}\NormalTok{, }\DecValTok{10}\NormalTok{)}
\end{Highlighting}
\end{Shaded}

\begin{verbatim}
#> [1]  1 10
#> [1] 1e+00 1e+10
\end{verbatim}

\hypertarget{dot-dot-dot}{%
\section{Dot dot dot}\label{dot-dot-dot}}

\begin{Shaded}
\begin{Highlighting}[]
\NormalTok{foo }\OtherTok{=} \ControlFlowTok{function}\NormalTok{ (a, b) \{}
  \FunctionTok{return}\NormalTok{ (a }\SpecialCharTok{/}\NormalTok{ b)}
\NormalTok{\}}
\NormalTok{bar }\OtherTok{=} \ControlFlowTok{function}\NormalTok{(a, ...) \{}
  \FunctionTok{return}\NormalTok{(}\FunctionTok{foo}\NormalTok{(a, ...))}
\NormalTok{\}}
\FunctionTok{bar}\NormalTok{(}\DecValTok{50}\NormalTok{, }\DecValTok{10}\NormalTok{)}
\end{Highlighting}
\end{Shaded}

\begin{verbatim}
#> [1] 5
\end{verbatim}

\begin{Shaded}
\begin{Highlighting}[]
\FunctionTok{bar}\NormalTok{(}\AttributeTok{b =} \DecValTok{10}\NormalTok{, }\DecValTok{50}\NormalTok{) }\CommentTok{\# named works too}
\end{Highlighting}
\end{Shaded}

\begin{verbatim}
#> [1] 5
\end{verbatim}

\hypertarget{operators}{%
\section{Operators}\label{operators}}

Operators like \texttt{+} or \texttt{-} or even \texttt{{[}} are all functions. To use them like a function, surround them with `.

\begin{Shaded}
\begin{Highlighting}[]
\StringTok{\textasciigrave{}}\AttributeTok{+}\StringTok{\textasciigrave{}}\NormalTok{(}\DecValTok{3}\NormalTok{, }\DecValTok{4}\NormalTok{)}
\end{Highlighting}
\end{Shaded}

\begin{verbatim}
#> [1] 7
\end{verbatim}

\begin{Shaded}
\begin{Highlighting}[]
\StringTok{\textasciigrave{}}\AttributeTok{*}\StringTok{\textasciigrave{}}\NormalTok{(}\DecValTok{3}\NormalTok{, }\DecValTok{4}\NormalTok{)}
\end{Highlighting}
\end{Shaded}

\begin{verbatim}
#> [1] 12
\end{verbatim}

\begin{Shaded}
\begin{Highlighting}[]
\StringTok{\textasciigrave{}}\AttributeTok{[}\StringTok{\textasciigrave{}}\NormalTok{(}\DecValTok{5}\SpecialCharTok{:}\DecValTok{10}\NormalTok{, }\DecValTok{2}\NormalTok{) }\CommentTok{\# you don\textquotesingle{}t need the close bracket (])}
\end{Highlighting}
\end{Shaded}

\begin{verbatim}
#> [1] 6
\end{verbatim}

\hypertarget{lists}{%
\chapter{Lists}\label{lists}}

A list is like an array but it can multiple types of elements.

\hypertarget{make-a-list}{%
\section{Make a list}\label{make-a-list}}

\begin{Shaded}
\begin{Highlighting}[]
\NormalTok{x }\OtherTok{=} \FunctionTok{list}\NormalTok{(}
  \AttributeTok{a =} \DecValTok{5}\NormalTok{,}
  \AttributeTok{b =} \DecValTok{2}\NormalTok{,}
  \AttributeTok{Long\_Name =} \FloatTok{4.8}\NormalTok{,}
  \StringTok{"named with spaces"} \OtherTok{=} \DecValTok{0}\NormalTok{,}
  \DecValTok{12}\NormalTok{, }\CommentTok{\# not every element needs a name}
  \AttributeTok{a =} \DecValTok{20} \CommentTok{\# names don\textquotesingle{}t have to be unique (but you really should avoid this)}
\NormalTok{)}
\end{Highlighting}
\end{Shaded}

\hypertarget{accessing-elements-in-a-list}{%
\section{Accessing elements in a list}\label{accessing-elements-in-a-list}}

Get a tuple of the key and value

\begin{Shaded}
\begin{Highlighting}[]
\NormalTok{x[}\StringTok{\textquotesingle{}a\textquotesingle{}}\NormalTok{] }\CommentTok{\# by key name}
\end{Highlighting}
\end{Shaded}

\begin{verbatim}
#> $a
#> [1] 5
\end{verbatim}

\begin{Shaded}
\begin{Highlighting}[]
\NormalTok{x[}\DecValTok{1}\NormalTok{]   }\CommentTok{\# by index}
\end{Highlighting}
\end{Shaded}

\begin{verbatim}
#> $a
#> [1] 5
\end{verbatim}

Multiple keys

\begin{Shaded}
\begin{Highlighting}[]
\NormalTok{x[}\FunctionTok{c}\NormalTok{(}\StringTok{\textquotesingle{}b\textquotesingle{}}\NormalTok{, }\StringTok{\textquotesingle{}a\textquotesingle{}}\NormalTok{)] }\CommentTok{\# by key name}
\end{Highlighting}
\end{Shaded}

\begin{verbatim}
#> $b
#> [1] 2
#> 
#> $a
#> [1] 5
\end{verbatim}

\begin{Shaded}
\begin{Highlighting}[]
\NormalTok{x[}\FunctionTok{c}\NormalTok{(}\StringTok{\textquotesingle{}b\textquotesingle{}}\NormalTok{, }\StringTok{\textquotesingle{}a\textquotesingle{}}\NormalTok{)] }\CommentTok{\# by key name}
\end{Highlighting}
\end{Shaded}

\begin{verbatim}
#> $b
#> [1] 2
#> 
#> $a
#> [1] 5
\end{verbatim}

Type the list name, then \texttt{\$}, and press tab. R will pop up a list of keys to autocomplete.
R uses \texttt{\$} in the way that other languages use \texttt{.}

\begin{Shaded}
\begin{Highlighting}[]
\NormalTok{x}\SpecialCharTok{$}\NormalTok{Long\_Name}
\end{Highlighting}
\end{Shaded}

\begin{verbatim}
#> [1] 4.8
\end{verbatim}

\emph{Note: Only the value is returned.}

\hypertarget{brackets-for-real}{%
\section{Brackets for real}\label{brackets-for-real}}

Sometimes, R will return the whole list or object even though you asked for just one element. So you need to use double brackets. Why? Because R is snarky and doesn't believe you actually want what you said. So you need to use double brackets to explain to R that you're sure this is what you want.

Double brackets only works for single items, not subsetting.

\begin{Shaded}
\begin{Highlighting}[]
\NormalTok{x[[}\StringTok{\textquotesingle{}a\textquotesingle{}}\NormalTok{]] }\CommentTok{\# by key name}
\end{Highlighting}
\end{Shaded}

\begin{verbatim}
#> [1] 5
\end{verbatim}

\begin{Shaded}
\begin{Highlighting}[]
\NormalTok{x[[}\DecValTok{1}\NormalTok{]]   }\CommentTok{\# by index}
\end{Highlighting}
\end{Shaded}

\begin{verbatim}
#> [1] 5
\end{verbatim}

\hypertarget{names-and-values}{%
\section{Names and values}\label{names-and-values}}

Use the \texttt{names()} function to get and set names. It behaves like an array.

\begin{Shaded}
\begin{Highlighting}[]
\FunctionTok{names}\NormalTok{(x)}
\end{Highlighting}
\end{Shaded}

\begin{verbatim}
#> [1] "a"                 "b"                 "Long_Name"        
#> [4] "named with spaces" ""                  "a"
\end{verbatim}

\begin{Shaded}
\begin{Highlighting}[]
\FunctionTok{names}\NormalTok{(x)[}\DecValTok{3}\NormalTok{]}
\end{Highlighting}
\end{Shaded}

\begin{verbatim}
#> [1] "Long_Name"
\end{verbatim}

You can modify names by assigning strings to the names function. This is weird. Take a minute to let it sink in.

\begin{Shaded}
\begin{Highlighting}[]
\FunctionTok{names}\NormalTok{(x) }\OtherTok{=} \FunctionTok{c}\NormalTok{(}\StringTok{"first"}\NormalTok{, }\StringTok{"second"}\NormalTok{, }\StringTok{"third"}\NormalTok{, }\StringTok{"fourth"}\NormalTok{)}
\FunctionTok{names}\NormalTok{(x)[}\DecValTok{3}\NormalTok{] }\OtherTok{=} \StringTok{"new name"}
\end{Highlighting}
\end{Shaded}

If all elements are the same type, this will get a vector of values

\begin{Shaded}
\begin{Highlighting}[]
\NormalTok{myList }\OtherTok{=} \FunctionTok{list}\NormalTok{(}\AttributeTok{a=}\DecValTok{1}\NormalTok{, }\AttributeTok{b=}\DecValTok{2}\NormalTok{, }\AttributeTok{c=}\DecValTok{3}\NormalTok{, }\AttributeTok{d=}\DecValTok{4}\NormalTok{)}
\FunctionTok{as.vector}\NormalTok{(}\FunctionTok{unlist}\NormalTok{(myList))}
\end{Highlighting}
\end{Shaded}

\begin{verbatim}
#> [1] 1 2 3 4
\end{verbatim}

\hypertarget{libraries-and-packages}{%
\chapter{Libraries and packages}\label{libraries-and-packages}}

A library or package is a collection of variables, datasets, functions, and/or operators.

It's called a ``package'' when being installed \texttt{install.packages("tidyverse")} and a ``library'' when being loaded for use \texttt{library(tidyverse)}.

A library and a package are the same thing, but R people will insist there is a difference. Whenever talking to R people, you've got a 50-50 chance of getting it right. If you get it wrong, you're going to get a short lecture. Just nod, and say ``yes, that makes sense, and the distinction is clearly important''. If you say anything else, you'll get a much longer more boring lecture.

I define these functions, so I don't have to worry about confusing the two.

\begin{Shaded}
\begin{Highlighting}[]
\NormalTok{install.library }\OtherTok{=}\NormalTok{ install.packages}
\NormalTok{package }\OtherTok{=}\NormalTok{ library}
\end{Highlighting}
\end{Shaded}

If you only want to access one function or variable in a library without loading the whole thing, you can use \texttt{::}

\begin{Shaded}
\begin{Highlighting}[]
\NormalTok{dplyr}\SpecialCharTok{::}\NormalTok{band\_instruments}
\end{Highlighting}
\end{Shaded}

\begin{verbatim}
#> # A tibble: 3 x 2
#>   name  plays 
#>   <chr> <chr> 
#> 1 John  guitar
#> 2 Paul  bass  
#> 3 Keith guitar
\end{verbatim}

\end{document}
